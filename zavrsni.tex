\documentclass[times, utf8, zavrsni, numeric]{fer}
\usepackage{booktabs}
\usepackage{import}
\usepackage{subfiles}
\usepackage{amsmath}
\usepackage{rtsched}
\usepackage{url}
\usepackage{graphicx}
\usepackage[utf8]{inputenc}
\usepackage[T1]{fontenc}
\usepackage{ragged2e}
%\usepackage[raggedright]{titlesec}

\usepackage{xcolor}
\usepackage{listings}
%\usepackage[nottoc]{tocbibind}

\usepackage{float}
\floatstyle{plaintop}
\restylefloat{table}

\definecolor{mGreen}{rgb}{0,0.6,0}
\definecolor{mGray}{rgb}{0.5,0.5,0.5}
\definecolor{mPurple}{rgb}{0.58,0,0.82}
\definecolor{backgroundColour}{rgb}{0.95,0.95,0.92}

\lstdefinestyle{CStyle}{
    backgroundcolor=\color{backgroundColour},   
    commentstyle=\color{mGreen},
    keywordstyle=\color{magenta},
    numberstyle=\tiny\color{mGray},
    stringstyle=\color{mPurple},
    columns=fixed,
    basicstyle=\ttfamily,
    breakatwhitespace=false,         
    breaklines=true,                 
    captionpos=b,                    
    keepspaces=true,                 
    numbers=left,                    
    numbersep=5pt,                  
    showspaces=false,                
    showstringspaces=false,
    showtabs=false,                  
    tabsize=4,
    language=C
}

\renewcommand\lstlistingname{Odsječak koda}
\renewcommand{\figurename}{Slika}
\renewcommand{\tablename}{Tablica}
\renewcommand\bibname{Literatura}

\begin{document}

% TODO: Navedite broj rada.
\thesisnumber{000}

% TODO: Navedite naslov rada.
\title{Nadogradnja operacijskog sustava FreeRTOS za primjenu u kontrolnim aplikacijama}

% TODO: Navedite svoje ime i prezime.
\author{Luka Jengić}

\maketitle

% Dodavanje zahvale ili prazne stranice. Ako ne želite dodati zahvalu, naredbu ostavite radi prazne stranice.
\zahvala{ Posebna zahvala mag.ing. Karli Salamun za brojne savjete i pomoć u izradi ovog rada.}

\renewcommand\contentsname{Sadržaj}
\tableofcontents

\chapter{Uvod} 
\subfile{tekstovi/uvod}

\chapter{Operacijski sustavi za rad u stvarnom vremenu}
\subfile{tekstovi/osnove_teorije_rasporedivanja}

\chapter{Modifikacija jezgre FreeRTOS-a}
\subfile{tekstovi/modifikacija_jezgre_FreeRTOS-a}

\chapter{Implementacija simulatora}
\subfile{tekstovi/implementacija_simulatora}

\chapter{Rezultati}
\subfile{tekstovi/rezultati}

\chapter{Zaključak}
\subfile{tekstovi/zakljucak}

%\nocite{*}

\bibliography{literatura}
\bibliographystyle{fer}

\begin{sazetak}
U ovom radu opisana je modifikacija operacijskog sustava za rad u stvarnom vremenu FreeRTOS.
U FreeRTOS je ugrađena podrška za kontrolu periodičnih zadataka. Korišten je model strogog sustava za rad u stvarnom vremenu 
s ublaženim uvjetima te strategija prekidanja poslova.
Nadalje implementirani su različiti algoritmi za raspoređivanje zadataka. Algoritmi su ispitivani i uspoređivani na velikom 
broju generiranih skupova zadataka. Za potrebe ovog rada, razvijen je generator testnih skupova zadataka. 
Dobiveni rezultati su u skladu s očekivanja i poklapaju se s iznesenom teroijom. 

\kljucnerijeci{sustavi za rad u stvarnom vremenu, FreeRTOS, raspoređivanje zadataka, \textit{skip-over model}, strategija prekidanja poslova.}
\end{sazetak}

% TODO: Navedite naslov na engleskom jeziku.
%\engtitle{Modification of FreeRTOS for Control Applications }
%\begin{abstract}

%This thesis describes modification of FreeRTOS real-time operating system.


%\keywords{Real time systems, FreeRTOS, task scheduling, skip-over model, job killing.}
%\end{abstract}

\end{document}
