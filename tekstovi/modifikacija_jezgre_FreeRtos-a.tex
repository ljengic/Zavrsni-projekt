\documentclass[../zavrsni.tex]{subfiles}

\begin{document}

\section{Operacijski sustav FreeRTOS}

FreeRTOS je operacijski sustav za rad u stvarnom vremenu otvorenog koda (eng. open source).U ovom odjeljku bit će objašnjena 
njegova implmenetacija i izvedba raspoređivanja zadataka. Kod FreeRTOS-a organiziran je u nekoliko datoteka i zauzima 
svega nekoliko kilobajta. Datoteka u kojoj je implementirano upravljanje zadatcima i njihovim raspoređivanjem je tasks.c.

Zadatci u FreeRTOS-u su upravljani kroz strukturu za upravljanje zadatcima (eng. TCB - Task Control Block). U toj strukturi nalaze
se sve informacije koje opisuju pojedini zadatak. Pri stvaranju novog zadatka u memoriji se alocira prostor za novu inačicu 
ove strukture u koju se potom upisuju parametri novokreiranog zadatka.
Zadatci su nadalje smješteni u različite liste, ovisno o tome u kojem stanju se zadatak nalazi. Postoje liste za zadatke koji su
spremni za izvršavanje, za zadatke koji su blokirani te za zadatke koji čekaju određeno vrijeme kako bi se ponovo mogli izvršavati.
U TCB-u svakog zadatka nalazi se informacija kojoj listi zadatak u danom trenutku pripada. Upravljanje i rad listi ostvaren je strukturama
ListItem\_t i List\_t. ListItem\_t je struktura koja sadrži podatke o pojedinom članu liste (pokazivači na prethodni i slijedeci ListItem\_t te 
pokazivač na TCB zadatka na kojeg se odnosi). U List\_t spremljeni su podatci o listi (broj elemenata, pokazivač na trenutni element te
pokazivač na kraj liste).


Zadatci koji su spremni za izvršavanje podjeljeni su u različite liste ovisno o tome kolikog su prioriteta (postoji posebna lista
 za sve moguće prioritete zadataka).
Stoga se pri raspoređivanju zadataka naprije pronađe lista najvišeg prioriteta koja nije prazna. Zadatci iz te liste dijele 
procesorsko vrijeme na način da se svakom zadatku iz liste dodjeljuje jedan vremenski odsječak procesorskog vremena (eng. tick).
Nakon što se zadatak izvede u jednom odsječku, vraća se na kraj liste čekanja. Ovaj način raspoređivanja zadataka naziva se dijeljenje
procesorskog vremena među zadatcima (eng. Time Slicing).

FreeRTOS se konfigurira putem FreeRTOSconfig.h datoteke i korisnik treba mjenjati samo tu datoteku. U toj datoteci nalaze se 
sve konstante preko kojih se uključuju pojedine funkcionalnosti ili postavljaju konstante bitne za rad sustva.  

Podrška za periodične zadatke nije implementirana u FreeRTOS-u te ju je potrebno dodati.
Također navedni pristup ne rješava problem preopterećenja sustava te je potrebno modificirati jezgru FreeRTOS-a kako bi se omogučilo 
predvidivo ponašanje ukoliko sustav uđe u trajno preopterećenje.Potrebne modifikacije detaljno su opisane u narednim podglavljima.

\section{Programska potpora za kontrolu izvršavanja periodičnih zadataka}

Uključenje funkcionalnosti za podršku periodičnim zadatcima ostvareno je putem konstante unutar FreeRTOSconfig.h datoteke.
Svi djelovi programskog koda zaduženi za periodične zadatke pisani su u odsječcima koji se kompajliraju samo ako je korištenje
periodičnih zadataka uključeno. Navedeno je ostvareno pomoću pretprocesorske naredbe \#if. Time je korisniku omogučeno uključenje
programske potpore za periodičke zadatke na jednostavan i brz način.
\begin{lstlisting}[style=CStyle,caption={Pretprocesorska naredba za uključenje periodičnih zadataka},captionpos=b]
#if ( configUSE_PERIODIC_TASK == 1 )

#endif
\end{lstlisting}

\subsection{Stvaranje periodičnih zadataka}

Prvi korak pri implementaciji programske podrške za izvršavanje periodičnih zadataka je proširenje TCB-a veličinama koje opisuju 
periodičan zadatak. Ovdje su definirane sve veličine bitne za kontrolu periodičkih zadataka koje su opisane u ranijim poglavljima.

\begin{lstlisting}[style=CStyle,caption={Varijable dodane u strukturu za kontrolu zadataka},captionpos=b]
#if ( configUSE_PERIODIC_TASK == 1 )
    uint8_t xTaskId;
    TickType_t xTaskPeriod;
    TickType_t start_time;
    TickType_t xTaskDuration;
    TickType_t xDeadline;
    TickType_t xRemainingTicks;
    
    // variables used for weakly hard conditions control
    uint8_t weakly_hard_constraint;
    uint8_t previous_deadline_met;
#endif
\end{lstlisting}

Nadalje napisana je funkcija xTaskCreatePeriodic koja se koristit za stvaranje periodičnih zadataka.
Navedena funkcija je proširenje funkcije xTaskCreate s novim varijablama potrebnim za opis periodičnih zadataka.

\begin{lstlisting}[style=CStyle,caption={Prototip funkcije xTaskCreatePeriodic},captionpos=b]
BaseType_t xTaskCreatePeriodic( TaskFunction_t pxTaskCode,
                        uint8_t id,
                        const char * const pcName, 
                        const configSTACK_DEPTH_TYPE usStackDepth,
                        void * const pvParameters,
                        UBaseType_t uxPriority,
                        TaskHandle_t * const pxCreatedTask,
                        TickType_t period,
                        TickType_t duration,
                        int weakly_hard_constraint);
\end{lstlisting} 

\subsection{Kontrola izvršenja priodičnih zadataka}

Izvršavanje zadataka u SRSV-ovima podjeljeno je u vremenske odsječke. U određenim vremenskim intervalima prekida se izvođenje poslova i 
određuje se koji posao se treba dalje izvršavati. U konkretnom slučaju simulacije na posix sustavu taj interval iznosi 1 milisekundu.
U FreeRTOS-u navedena funkcionalnost implementrana je u xTaskIncrementTick funkciji. Prekidni sustav periodički poziva navedenu funkciju 
i u njoj je potrebno dodati funkcionalnost kontrole periodičnih zadataka.

Za potrebe ovog projekta zadatci su spremani u dvije liste. Jedna lista u kojoj se čuvaju zadatci spremni za izvršavanje, ranije 
implementirana u FreeRTOS-u i novododana lista nazvana xWaitTaskList u kojoj su zadatci koji su na čekanju. Za dodavanje zadataka u liste
napisane su dvije funkcije, za jedna po svakoj listi.

\begin{lstlisting}[style=CStyle,caption={Definicije funkcija za dodavanje zadataka u opisane liste},captionpos=b]
void addTaskToReadyList(TCB_t * const pxItemToAdd);
void addTaskToWaitList(TCB_t * const pxItemToRemove);
\end{lstlisting}

Kontrola izvršavanja u ovom projektu realizirana je apsolutno na višekratnike perioda zadataka od početka simulacije.

U svakom pozivu funkcije xTaskIncrementTick potrebno je proći po svim zadatcima u obje liste. Zadatci koji su spremni za izvršavanje
možda se nebi uspjeli izvršiti do roka izvršenja čak kad bi dobili svo procesorsko vrijeme. Pošto nema koristi od posla koji se
djelomično izvršio takve zadatke potrebno je odmah smjestiti u listu za čekanje i detektirati njihovo propuštanje roka izvršenja.
Ova strategija zove se prekidanje poslova koji se ne mogu izvršiti do svog krajnjeg roka izvršenja(eng. job killing). Time se
podiže kvaliteta usluge, jer je osigurano da poslovi koji se nikako neće izvršiti na vrijeme ne zauzimaju procesorsko vrijeme 
ostalim zadatcima. Funkcionalnost prekidanja poslova implementirana je u funkciji killTasks. U njoj prolazimo po svim zadatcima koji se 
nalaze u listi zadataka spremnih za izvršavanje i na temelju podataka o trenutnom stanju posla odlučujemo treba li ga prekinuti ili ne.
U nastavku je priložen navedeni uvjet i graf koji opisuje navedenu situaciju prekidanja zadataka.
\begin{lstlisting}[style=CStyle,caption={Uvjet za prekidanje izvođenja zadatka},captionpos=b]
if((xTaskGetTickCount() - Tcb->start_time + Tcb->xRemainingTicks) > 
Tcb->xTaskPeriod)
\end{lstlisting}

Opisana situacija prikazana je na vremenskom dijagramu 3.1. U primjeru imamo 3 zadatka, od kojih je zadatak broj 1 većeg prioriteta dok su
preostala dva zadatka manjeg prioriteta. Periodi zadataka su slijedno 4, 8, i 4, dok su vremena izvršavanja 1, 5, i 2. (dodaj ovo ispod slike
u figure radije, bit ce bolje i preglednije)
Vidimo da je zadtak broj 2 potrošio 4 vremenska odječka procesorskog vremena, a na kraju se nije uspio izvršiti na vrijeme.
Strategija prekidanja poslova uvodi provjeru svaki vremenski odječak može li se zadatak izvesti do roka. U navedenom primjeru u 
trenutku t = 5 zadatak bi bio prekinut. Tada bi raspored poslova izgledao kao na dijagramu 3.2. Treba primjetiti da se uz prekidanje
poslova povećala kvaliteta usluge jer se je zadatak broj 3 sad uspio izvršiti u oba perioda. Važna napomena uz navedeni primjer je
da služi samo za objašnjenje strategije prekidanja poslova te je stoga pretpostavljeno da su svi zadatci jednako kritični.  

\begin{figure}[h]
    \centering

    \begin{RTGrid}[width=7cm]{3}{8}

      \TaskArrDead{1}{0}{4}     
      \TaskArrDead{1}{4}{4}
  
      \TaskExecution{1}{0}{1}
      \TaskExecution{1}{4}{5}

      \TaskArrDead{2}{0}{8}     

      \TaskExecution[color=red]{2}{3}{4}
      \TaskExecution[color=red]{2}{5}{8}

      \TaskArrDead{3}{0}{4}     
      \TaskArrDead{3}{4}{4}
  
      \TaskExecution{3}{1}{3}


    \end{RTGrid}

    \caption{Primjer rasporeda bez prekidanja zadataka}
    \label{fig:ex1}
  \end{figure}

  \begin{figure}[h]
    \centering

    \begin{RTGrid}[width=7cm]{3}{8}

      \TaskArrDead{1}{0}{4}     
      \TaskArrDead{1}{4}{4}
  
      \TaskExecution{1}{0}{1}
      \TaskExecution{1}{4}{5}

      \TaskArrDead{2}{0}{8}     

      \TaskExecution[color=red]{2}{3}{4}

      \TaskArrDead{3}{0}{4}     
      \TaskArrDead{3}{4}{4}
  
      \TaskExecution{3}{1}{3}
      \TaskExecution{3}{5}{7}


    \end{RTGrid}

    \caption{Primjer rasporeda s prekidanjem zadataka}
    \label{fig:ex1}
  \end{figure}

Nadalje, u funkciji wakeTasks provjerava se trebaju li se zadatci u listi za čekanje prebaciti u listu zadataka spremih za izvršavanje.
Ukoliko je zadatak u listi za čekanje i ukoliko se program nalazi na višekratniku njegova perioda, zadatak se dodaju u listu 
zadataka spremih za izvršavanje. Pri tome je potrebno u varijablu start\_time upisati trenutno vrijeme (trenutak u kojem se zadatak 
krenio izvršavati). Vrijeme početka izvršavanja je važno za provjeru treba li zadatak prekinuti i je li se izvršio do krajnjeg roka 
završetka.

Varijabla xRemainingTicks u svakom trenutku pamti koliko je vremenskih odsječaka ostalo poslu da se do kraja izvrši. 
Svaki put kada se posao izvršava ta varijabla se umanjuje za 1. Ukoliko se vrijednost xRemainingTicks smanji na 0, 
posao je gotov i zabilježava se njegovo pravovremeno izvršavanje.

\section{Implementacija algoritama za raspoređivanje zadataka}

Algoritmi za raspoređivanje zadataka podrazumijevaju logiku kojom se odabire koji zadatak će se idući poslati na izvršavanje.
Liste implementirane u FreeRTOS-u imaju mogućnost sortiranja po vrijednosti varijable xItemValue koju sadrži svaki član liste 
(u ListItem\_t strukturi). To svojstvo je iskorišteno kako bi zadatke koji su spremni za izvršavanje poredali po željenom redosljedu.



\end{document}