\documentclass[../zavrsni.tex]{subfiles}

\begin{document}

\section{Operacijski sustav FreeRTOS}

FreeRTOS je operacijski sustav za rad u stvarnom vremenu otvorenog koda (eng. open source) namjenjen primjeni u ugradbenim 
račualnim sustavima. U ovom potpoglavlju bit će objašnjena 
njegova implmenetacija i izvedba raspoređivanja zadataka. Kod FreeRTOS-a organiziran je u nekoliko datoteka i zauzima 
svega nekoliko kilobajta. Datoteka u kojoj je implementirano upravljanje zadatcima i njihovim raspoređivanjem je \texttt{tasks.c}.

Zadatci u FreeRTOS-u su opisani strukturom za upravljanje zadatcima (eng. TCB - Task Control Block). U toj strukturi nalaze
se sve informacije koje opisuju pojedini zadatak (ime zadatka, prioritet,broj zadatka...).  
Zadati se stvaraju funkcijom \texttt{xTaskCreate()}.
Pri stvaranju novog zadatka u alocira se prostor u memoriji za novu inačicu 
ove strukture u koju se potom upisuju parametri novokreiranog zadatka.

Zadatci su smješteni u različite liste, ovisno o tome u kojem stanju se zadatak nalazi. Postoje liste za zadatke koji su
spremni za izvršavanje, za zadatke koji su blokirani te za zadatke koji čekaju određeno vrijeme kako bi se ponovo mogli izvršavati.
U TCB-u svakog zadatka nalazi se informacija kojoj listi zadatak u danom trenutku pripada. Upravljanje listama zadataka ostvareno je strukturama
\texttt{ListItem\_t} i \texttt{List\_t}. \texttt{ListItem\_t} je struktura koja sadrži podatke o pojedinom članu liste (pokazivači na prethodni i 
sljedeći \texttt{ListItem\_t} te 
pokazivač na TCB zadatka na kojeg se odnosi). U \texttt{List\_t} spremljeni su podatci o listi (broj elemenata, pokazivač na trenutni element te
pokazivač na kraj liste).

Informacija o tome koji zadatak se trenutno izvršava pohranjena je u pokazivaču na TCB strukturu \texttt{pxCurrentTCB}.
Zadatci koji su spremni za izvršavanje podijeljeni su u različite liste ovisno o tome kojeg su prioriteta (postoji posebna lista
 za sve razine prioriteta zadataka).
Stoga se pri raspoređivanju zadataka naprije pronađe lista najvišeg prioriteta koja nije prazna. Zadatci iz te liste dijele 
procesorsko vrijeme na način da se svakom zadatku iz liste dodjeljuje jedan vremenski odsječak procesorskog vremena (eng. tick).
Nakon što se zadatak izvede u jednom odsječku, vraća se na kraj liste čekanja. Ovaj način raspoređivanja zadataka naziva se dijeljenje
procesorskog vremena među zadatcima (eng. Time Slicing).

FreeRTOS se konfigurira putem datoteke \texttt{FreeRTOSconfig.h} i korisnik treba mijenjati samo tu datoteku. U toj datoteci nalaze se 
sve konstante preko kojih se uključuju pojedine funkcionalnosti ili postavljaju konstante bitne za rad sustva.  

Podrška za upravljanje zadatcima u FreeRTOS-u ne podržava definiranje i kontrolu periodičkih zadataka, stoga je potrebno proširiti podsustav 
za upravljanje zadatvima u FreeRTOS-u.
Također navedni pristup ne rješava problem preopterećenja sustava te je potrebno modificirati jezgru FreeRTOS-a kako bi se omogučilo 
predvidivo ponašanje ukoliko sustav uđe u trajno preopterećenje. Potrebne modifikacije detaljno su opisane u narednim podglavljima.

\section{Programska potpora za kontrolu izvršavanja periodičnih zadataka}

Uključenje funkcionalnosti za upravljanje periodičnim zadatcima ostvareno je putem konstante \texttt{configUSE\_PERIODIC\_TASK} dodane unutar 
datoteke \texttt{FreeRTOSconfig.h}. Korisnik postavljanjem navedene konstante u 1 na brz i jednostavan način uključuje podršku za periodične zadatke.
Svi djelovi programskog koda zaduženi za periodične zadatke pisani su u odsječcima koji se uključuju u proces prevođenja samo ako je vrijednost 
\texttt{configUSE\_PERIODIC\_TASK} jednaka 1. Pritom je korištena pretprocesorska naredba \#if. 
\begin{lstlisting}[style=CStyle,caption={Pretprocesorska naredba za uključenje periodičnih zadataka},captionpos=b]
#if ( configUSE_PERIODIC_TASK == 1 )

#endif
\end{lstlisting}

\subsection{Stvaranje periodičnih zadataka}

Prvi korak pri implementaciji programske podrške za izvršavanje periodičnih zadataka je proširenje TCB-a veličinama koje opisuju 
periodičan zadatak. Ovdje su definirane sve veličine bitne za kontrolu periodičkih zadataka koje su opisane u ranijim poglavljima.

\begin{lstlisting}[style=CStyle,caption={Varijable dodane u strukturu za kontrolu zadataka},captionpos=b]
#if ( configUSE_PERIODIC_TASK == 1 )
  // variables used for periodic task control    
  uint8_t xTaskId;
  TickType_t xTaskPeriod;
  TickType_t start_time;
  TickType_t xTaskDuration;
  TickType_t xDeadline;
  TickType_t xRemainingTicks;
#endif
\end{lstlisting}

Nadalje, napisana je funkcija \texttt{xTaskCreatePeriodic()} koja se koristi za stvaranje i inicijalizaciju periodičnih zadataka.
Navedena funkcija je proširenje \texttt{funkcije xTaskCreate()} s novim varijablama potrebnim za opis periodičnih zadataka.

\begin{lstlisting}[style=CStyle,caption={Prototip funkcije xTaskCreatePeriodic},captionpos=b]
BaseType_t xTaskCreatePeriodic( TaskFunction_t pxTaskCode,
                        uint8_t id,
                        const char * const pcName, 
                        const configSTACK_DEPTH_TYPE usStackDepth,
                        void * const pvParameters,
                        UBaseType_t uxPriority,
                        TaskHandle_t * const pxCreatedTask,
                        TickType_t period,
                        TickType_t duration,
                        int weakly_hard_constraint);
\end{lstlisting} 

\subsection{Kontrola izvršenja priodičnih zadataka}

Izvršavanje zadataka u SRSV-ovima podjeljeno je u vremenske odsječke. U određenim vremenskim intervalima (eng. ticks) prekida se izvođenje poslova i 
određuje se koji posao se treba dalje izvršavati.
U FreeRTOS-u navedena funkcionalnost implementrana je u \texttt{xTaskIncrementTick()} funkciji. Prekidni sustav periodički poziva navedenu funkciju 
i u njoj je potrebno dodati funkcionalnost kontrole periodičnih zadataka.

Kontrola izvođenja periodičkih zadataka realizirana je pomoću varijable \texttt{xRemainingTicks} koja u svakom trenutku pamti 
koliko je vremenskih odsječaka ostalo do potpunog izvršenja posla.
Svaki vremenski odsječak u kojem se posao izvršava ta varijabla se umanjuje za 1. Ukoliko se vrijednost \texttt{xRemainingTicks} smanji na 0, 
posao je gotov i zabilježava se njegovo pravovremeno izvršavanje. Svaki puta kada se posao kreće izvršavati, varijabla 
\texttt{xRemainingTicks} postavlja se na vrijeme njegovog izvršavanja.

Za potrebe ovog projekta zadatci su spremani u dvije liste. Jedna lista u kojoj se čuvaju zadatci spremni za izvršavanje, ranije 
implementirana u FreeRTOS-u i novododana lista nazvana \texttt{xWaitTaskList} u kojoj su zadatci koji su na čekanju. Za dodavanje zadataka u liste
napisane su dvije funkcije, za jedna po svakoj listi.

\begin{lstlisting}[style=CStyle,caption={Deklaracije funkcija za dodavanje zadataka u opisane liste},captionpos=b]
void addTaskToReadyList(TCB_t * const pxItemToAdd);
void addTaskToWaitList(TCB_t * const pxItemToRemove);
\end{lstlisting}

%U svakom pozivu funkcije \texttt{xTaskIncrementTick()} potrebno je proći po svim zadatcima u obje liste. Zadatci koji su spremni za izvršavanje
%možda se ne bi uspjeli izvršiti do roka izvršenja čak kad bi dobili svo procesorsko vrijeme. 

Vraćanje zadataka iz liste za čekanje u listu zadataka spremnih za izvršavanje implementirano je u funkciji \texttt{wakeTasks()}.
U njoj se petljom iterira po svim zadatcima koji su u stanju čekanja i za svakog se provjerava treba li ga prebaciti u listu zadataka 
spremih za izvršavanje.
Ukoliko je zadatak u listi za čekanje i ukoliko se program nalazi na višekratniku njegova perioda, zadatak se dodaju u listu 
zadataka spremih za izvršavanje. Pri tome je potrebno u varijablu \texttt{start\_time} upisati trenutno vrijeme (trenutak u kojemu je 
zadatak posao spreman za izvršavanje).Također moramo osvježiti vrijednost varijable texttt{xRemainingTicks}. 
Vrijeme početka izvršavanja je važno za provjeru treba li zadatak prekinuti i je li se izvršio do krajnjeg roka završetka 
što će detaljnije biti objašnjeno u tesktu koji slijedi.

%Kontrola izvršavanja u ovom projektu realizirana je relativno na višekratnike perioda zadataka od početka simulacije.

\section{Strategija prekidanja poslova}

Ponekad posao dođe na red za ivršavanje, ali čak i ako dobije svo procesorsko vrijeme ne stigne se izvršiti do krajnjeg roka za izvršavanje.
Kako kod strogih zadataka nema koristi od posla koji se djelomično izvršio, ukoliko posao dođe do krajnjeg roka za izvršavanje, a nije se izvršio
 treba prekinuti njegovo daljnje izvršavanje. Ova strategija zove se prekidanje poslova koji se ne mogu izvršiti do svog 
 krajnjeg roka izvršenja (engl. job killing). Time se podiže kvaliteta usluge, jer je osigurano da poslovi koji se nikako neće 
 izvršiti na vrijeme ne zauzimaju proesorsko vrijeme ostalim zadatcima. Strategiju prekidanja poslova dodatno možemo poboljšati
 ukoliko unaprijed prolazimo po svim spremnim poslovima i provjeravamo mogu li se izvršiti do roka ako im je na raspolaganju svo 
 procesorsko vrijeme od datog trenutka. Ako se posao nebi stigao obaviti treba ga odmah prebaciti u listu za čekanje i detektirati 
 propuštanje roka izvršavanja. Uz opisani pristup zadatci koji se ne mogu izvesti neće se niti kretati izvoditi, nego će samo
 biti detektirano njihovo propuštanje čime se dodatno poboljšava rasporedivost zadataka.

Opisana situacija prikazana je vremenskim dijagramima na slikama 3.1 i 3.2. U primjeru imamo 3 zadatka jednakih prioriteta. Radi jednostavnosti
je pretpostavljeno da su svi zadatci jednako kritični za sustav (ne koristi se model ublaženo-strogih uvjeta). Za raspoređivanje zadataka korišten je algoritam EDF. 
Periodi zadataka i vremena njihova izvršavanja dani su u tablici 3.1. Sustav je u stanju trajnog preopterećenja sa ukupnim
faktorom opterećenja 1.42.

\begin{table}[h!]
\begin{center}
  \begin{tabular}{||c || c c c||} 
   \hline
   Broj Zadatka & Period & Vrijeme izvršavanja & Faktor opterećenja \\ [0.5ex] 
   \hline\hline
   1 & 6 & 1 & 0.17 \\ 
   \hline
   2 & 8 & 6 & 0.75 \\
   \hline
   3 & 4 & 2 & 0.50 \\
   \hline
  \end{tabular}
\end{center}
\caption{\label{tab:table-name}Skup zadataka korišten u opisu strategije prekidanja poslova.}
\end{table}

Treba primjetiti da je bez korištenja strategije prekidanja poslova zadatak $\tau_2$ potrošio 13 vremenskih odječka procesorskog vremena, a 
na pritom se nije uspio niti jednom izvršiti na vrijeme.
Predložena modificirana strategija prekidanja poslova uvodi provjeru svaki vremenski odječak može li se zadatak izvesti do  krajnjeg roka 
za izvršavanje. 
U navedenom primjeru, u 
trenutku t = 4 sustav bi obavio provjeru može li se zadatak $\tau_2$ izvršiti na vrijeme, zaključio bi da ne može i nebi ga niti krenio 
izvoditi. Tada se oslobađa procesorsko vrijeme za zadatke $\tau_1$ i $\tau_3$ i
raspored poslova izgleda kao na dijagramu 3.2. 

Kako bi vidjeli kakav utjecaj na sustav ima prekidanje poslova za oba slučaja potrebno je izračunati kvalitetu usluge.
Bez korištenja strategije prekidanja poslova
izvršilo se 7 od ukupno 13 poslova. Kvaliteta usluge u tom slučaju je 0.54. Uz primjenu opisanog poboljšanja uspješno se izvelo 10 
poslova, što daje kvalitetu usluge 0.67. Iz prikazanog primjera može se zaključiti da korištenje prekidanja poslova u uvjetima 
preopterećenog sustava znatno podiže kvalitetu usluge i poboljšava rad sustava.

Važna napomena uz navedeni primjer je da služi samo za objašnjenje strategije prekidanja poslova te je stoga 
pretpostavljeno da su svi zadatci jednako kritični.  

\begin{figure}[h!]
  \centering

  \begin{RTGrid}[width=13cm]{3}{24}

      \TaskArrDead{1}{0}{6}     
      \TaskArrDead{1}{6}{6}
      \TaskArrDead{1}{12}{6}
      \TaskArrDead{1}{18}{6}
      \TaskArrival{1}{24}
  
      \TaskExecution{1}{2}{3}
      \TaskExecution{1}{11}{12}
      \TaskExecution{1}{22}{23}

      \TaskArrDead{2}{0}{8}     
      \TaskArrDead{2}{8}{8}
      \TaskArrDead{2}{16}{8}
      \TaskArrival{2}{24}
  
      \TaskExecution[color=red]{2}{3}{9}
      \TaskExecution[color=red]{2}{14}{20}
      \TaskExecution[color=red]{2}{23}{24}

      \TaskArrDead{3}{0}{4}     
      \TaskArrDead{3}{4}{4}
      \TaskArrDead{3}{8}{4}
      \TaskArrDead{3}{12}{4}
      \TaskArrDead{3}{16}{4}
      \TaskArrDead{3}{20}{4}
      \TaskArrival{3}{24}
  
      \TaskExecution{3}{0}{2}
      \TaskExecution{3}{9}{11}
      \TaskExecution{3}{12}{14}
      \TaskExecution{3}{20}{22}

  \end{RTGrid}

  \caption{Raspored zadataka bez prekidanja poslova}
  \label{fig:ex1}
\end{figure}

\begin{figure}[h!]
  \centering

  \begin{RTGrid}[width=13cm]{3}{24}

      \TaskArrDead{1}{0}{6}     
      \TaskArrDead{1}{6}{6}
      \TaskArrDead{1}{12}{6}
      \TaskArrDead{1}{18}{6}
      \TaskArrival{1}{24}
  
      \TaskExecution{1}{2}{3}
      \TaskExecution{1}{6}{7}
      \TaskExecution{1}{16}{17}
      \TaskExecution{1}{19}{20}

      \TaskArrDead{2}{0}{8}     
      \TaskArrDead{2}{8}{8}
      \TaskArrDead{2}{16}{8}
      \TaskArrival{2}{24}
  
      \TaskExecution{2}{10}{16}

      \TaskArrDead{3}{0}{4}     
      \TaskArrDead{3}{4}{4}
      \TaskArrDead{3}{8}{4}
      \TaskArrDead{3}{12}{4}
      \TaskArrDead{3}{16}{4}
      \TaskArrDead{3}{20}{4}
      \TaskArrival{3}{24}
  
      \TaskExecution{3}{0}{2}
      \TaskExecution{3}{4}{6}
      \TaskExecution{3}{8}{10}
      \TaskExecution{3}{17}{19}
      \TaskExecution{3}{20}{22}

  \end{RTGrid}

  \caption{Raspored zadataka sa primjenjnim prekidanjem poslova}
  \label{fig:ex1}
\end{figure}

Funkcionalnost prekidanja poslova implementirana je u funkciji \texttt{killTasks()} koja se poziva svaki vremenski 
odsječak procesorskog vremena. U njoj iteriramo po svim zadatcima koji se 
nalaze u listi zadataka spremnih za izvršavanje i na temelju podataka o trenutnom stanju posla odlučujemo treba li ga prekinuti ili ne.
Uvjet za prekid dobije se tako da usporedimo vrijeme u kojem bi posao bio obavljen sa krajnjim rokom za izvršavanje. Vremena su uspoređivana
relativno u odnosu na vrijeme kada je posao postao spreman za izvršavanje (višekratnik perida).
U nastavku je priložen navedeni uvjet.
\begin{lstlisting}[style=CStyle,caption={Uvjet za prekidanje izvođenja posla},captionpos=b]
if((xTaskGetTickCount() - TaskTcb->start_time + TaskTcb->xRemainingTicks) > 
TaskTcb->xTaskDeadline)
\end{lstlisting}

\section{Strogi sustav za rad u stvarnom vremenu s ublaženim uvjetima}

Kao što je ranije rečeno, implementiran je skip-over model SRSV-a s ublaženo-strogim uvjetima.
Kod njega na zadatke postavimo uvjet da se smije propustiti jedan posao u nekoliko slijednih perioda.
Navedeno je implementirano pomoću dvije cjelobrojne varijable dodane u TCB strukturu zadatka.

% vrste weakly hard uvjeta da ili ne
\begin{figure}[h]
\centering
\begin{lstlisting}[style=CStyle,caption={Varijable dodane u strukturu za kontrolu zadataka},captionpos=b]
#if ( configUSE_PERIODIC_TASK == 1 )
  // variables used for weakly hard conditions control
  uint8_t weakly_hard_constraint;
  uint8_t previous_deadline_met;
#endif
\end{lstlisting}
\end{figure}

Varijablom \texttt{weakly\_hard\_constraint} zadano je u koliko slijednih perioda se jedan posao smije propustiti.
Na primjer, ukoliko je zadana vrijednost ublaženo-strogog zadatka 5, to znači da se jedan posao smije propustiti unutar pet perioda 
koji slijede jedan za drugim.
Poseban slučaj su vrijednosti 0 i 1. Vrijednost 0 zadavat će se za najkritičnije zadatke koji se nikada ne smiju propustiti. 
U teoriji raspoređivanja, tom slučaju odgovara skip faktor vrijednosti beskonačno.
S druge strane, vrijednost 1 daje se manje bitnim zadatcima koji se uvijek mogu propustiti i čije neizvršavanje nije pogubno za sustav.
Postavljanjem ove vrijednosti točno se zadaje kako se poslovi smiju propuštati. Vrijednosti se određuju 
temeljem kritičnosti zadataka i posljedica koje nose propuštanja poslova, kako bi sustav nakon propuštanja poslova ostao 
stabilan i neoštećen.

Da bi sustav u svakom trenuku znao smije li propustiti neki posao potrebna je još jedna varijabla. To je varijabla 
\texttt{previous\_deadline\_met} u kojoj je upisan broj poslova koji su se prethodno pravovremeno izvršili jedan iza drugoga.
Svaki puta kada se posao izvrši na vrijeme ova varijabla se povečava za jedan, a kada se proputi izvršavanje posla ili ga se prekine tada 
se vrijednost varijable postavlja na 0.

Poznavajući vrijednosti ovih dviju varijabli u svakom trenutku sustav zna mora li se zadatak izvršiti ili ne. Posao se mora izvršiti
ako se prethodno zadatak izvršio manje puta od vrijednosti \texttt{weakly\_hard\_constraint} umanjene za 1. Ovaj uvjet bit će 
korišten prilikom implementacije raspoređivanja zadataka.

\begin{lstlisting}[style=CStyle,caption={Uvjet za slučaj kada se zadatak mora izvršiti},captionpos=b]
if( TaskTcb->previous_deadline_met < (TaskTcb->weakly_hard_constraint-1))
\end{lstlisting}

Zbog usporedbe različitih algoritama, na početku simulacije (pri stvaranju zadataka) varijabla \texttt{previous\_deadline\_met} će 
biti postavljena na 0. Pri raspoređivanju zadtaka to odgovara najgorem mogućem slučaju.

\section{Implementacija algoritama za raspoređivanje zadataka}

Algoritmi za raspoređivanje zadataka podrazumijevaju logiku kojom se odabire koji zadatak će se idući poslati na izvršavanje.
To je najvažniji dio SRSV-a jer o raspoređivanju zadataka ovisi hoće li sustav biti pouzdan i predvidljiv.

Liste implementirane u FreeRTOS-u imaju mogućnost sortiranja po vrijednosti varijable \texttt{xItemValue} koju sadrži svaki član liste 
(u strukturi \texttt{ListItem\_t}). To svojstvo je iskorišteno kako bi zadatke koji su spremni za izvršavanje poredali po željenom 
redosljedu. Element s vrha liste uvijek će biti onaj sa najmanjom vrijednosti \texttt{xItemValue}. Njega ćemo uzimati za izvršavanje 
 što znači da manja vrijednost \texttt{xItemValue} daje veći prioritet pri raspoređivanju.
pri izvršavanju. Vrijednost varijable \texttt{xItemValue} mijenja se putem macro funkcije \\ \texttt{listSET\_LIST\_ITEM\_VALUE()} 
implementirane u FreeRTOS-u.

Prvi korak u modifikaciji raspoređivača zadataka FreeRTOS-a bio je promjena macro-a \texttt{taskSELECT\_HIGHEST\_PRIORITY\_TASK()}.
U njemu se određuje koji zadatak će se idući poslati na izvršavanje iz liste koja trenutno sadrži zadatke najvećeg prioriteta.
Prije izmjene zadatci su dijelili procesorsko vrijeme, svaki zadatak po jedan vremenski odsječak. To je bilo realizirano pozivom
macro-a \texttt{listGET\_OWNER\_OF\_NEXT\_ENTRY} koji je iterirao po svim članovima liste. Umjesto toga, potrebno je svaki vremenski odsječak
dohvatiti prvi element liste. Za to u FreeRTOS-u postoji macro \texttt{listGET\_OWNER\_OF\_HEAD\_ENTRY} koji vrati pokazivač na 
prvi element liste. To će biti zadatak koji se treba izvršavati jer je lista sortitana u uzlaznom poretku. 
U svakom vremenskom odsječku u varijablu \texttt{pxCurrentTCB} treba pohraniti pokazivač na prvi element liste. Time je osigurano da
se uvijek izvodi zadatak najvećeg prioriteta, skroz dok se ne izvede, dok ga ne prekine neki kritičniji zadatak ili dok ga sustav 
ne prekine strategijom prekidanja poslova.

%\begin{figure}[h!]
%\centering
%\begin{lstlisting}[style=CStyle,caption={Primjer uključenja algoritma BWP},captionpos=b]
%pxCurrentTCB=listGET_OWNER_OF_HEAD_ENTRY(&( pxReadyTasksLists[ uxTopPriority ] ) );
%\end{lstlisting}
%\end{figure}

Nadalje je pri stavljanju zadatka u listu zadataka spremih za izvršavanje potrebno generirati i upisati vrijednost \texttt{xItemValue}
prema kojoj će se zadatci sortirati. To je implementirano u macro funkciji \texttt{prvAddTaskToReadyList}. Za svaki algoritam 
potrebno je imati posebno implementiranu funkciju \texttt{prvAddTaskToReadyList}. To je ostvareno preko konstanti u datoteci 
FreeRTOSconfig.h. Za svaki algoritam definirana je jedna konstanta, i koristi se onaj algoritam kojem je vrijednost konstante 1.
Prije pokretanja simulacije korisnik treba odabrati koji algoritam se koristi, vrijednost njegove konstante postaviti u 1, a svih 
ostalih u 0. Time je omogućeno da se macro \texttt{prvAddTaskToReadyList} napiše za svaki algoritam zasebno, ali unutar pretprocesorske
naredbe \#if, te će na kraju samo jedna verzija biti uključena u program. Ukoliko korisnik pokuša istovremeno uključiti više 
algoritama program će javiti grešku.

\begin{figure}[h]
\centering
\begin{lstlisting}[style=CStyle,caption={Primjer uključenja algoritma BWP},captionpos=b]
#define configEDF_ALGORITHM                        0
#define configRTO_ALGORITHM                        0
#define configBWP_ALGORITHM                        1
\end{lstlisting}
\end{figure}

\subsection{Implementacija algoritma EDF}

Kako je ranije opisano, algoritam EDF sortira zadatke prema vremenu krajnjeg roka za izvršavanje. Pri stavljanju zadatka u 
listu za spremne zadatke, potrebno je izračunati apsolutno vrijeme roka za izvršenje posla. Pošto su nam rokovi za izvršavanje
uvijek višekratnici perioda, zapravo trebamo izračunati vrijeme sljedećeg perioda. To radimo tako da od trenutnog vremena 
oduzmemo vrijeme proteklo od početka perioda (time smo dobili početak perioda) te još nadodamo jedan period. Kod EDF algoritma
u obzir ne uzimamo ublažene uvjete kod strogog SRSV-a, nego sve poslove promatramo jednako važnima.

ZAMJENI OVO SA NEKIM PSEUDO KODOM !!!
\begin{lstlisting}[style=CStyle,caption={Računanje \texttt{xItemValue} vrijednosti za EDF algoritam},captionpos=b]
listSET_LIST_ITEM_VALUE(&(( pxTCB )->xStateListItem),(( pxTCB )->xTaskPeriod+xTaskGetTickCount()-(xTaskGetTickCount()%(( pxTCB )->xTaskPeriod))));
\end{lstlisting}


\subsection{Implementacija algoritma RTO}

Kod algoritma RTO izvršavaju se samo zadatci koji se moraju izvršiti, a svi ostali se preskaču. Drugim rječima, čim neki posao 
možemo preskočii, preskočiti ćemo ga. Zadatci koji se šalju na izvršavanje raspoređuju se po EDF algoritmu. Jedina razlika u odnosu
na EDF algoritam je ta što provjeravamo ublažene uvjete u strogom SRSV-u te ukoliko se zadatak može prekočiti ne dodajemo ga u 
listu zadataka koji čekaju na izvršavanje, nego on ostaje u listi za čekanje. 

DODAJ PSEUDO KOD OVOG !!!!

\subsection{Implementacija algoritma BWP}

Kako je ranije opisano, BWP algoritam je proširenje RTO algoritma. Sada se svi zadatci stavljaju u listu zadataka spremnih za izvršavanje,
ali plavi zadatci na red dolaze tek kada su svi strogi poslovi obavljeni. To je potrebno osigurati sortiranjem prema vrijednosti
\texttt{xItemValue}. Svi poslovi i dalje će biti raspoređivani EDF algoritmom, ali podjeljeni u dvije skupine (poslovi koji se moraju
izvršiti i oni čije je izvršavanje opcionalno). Pošto je \texttt{xItemValue} je 32-bitni cijeli broj, a manja vrijednost daje prioritet
za izvršavanje, plavim poslovima najznačajniji bit je postavljen u 1. Time je osigurano da takvi poslovi budu 
zadnji na redu za izvršavanje jer će im \texttt{xItemValue} biti najveći.  

DODAJ PSEUDO KOD OVOG !!!!

\end{document}