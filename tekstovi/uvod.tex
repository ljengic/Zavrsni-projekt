\documentclass[../zavrsni.tex]{subfiles}

\begin{document}

\sloppy

\justifying

Sustavi za rad u stvarnom vremenu (engl. \textit{real-time systems}) sustavi su u kojima nije bitan samo
rezultat operacije, nego je jednako važna i pravovremenost njezina izvršenja. Najvažnija svojstva koja takvi sustavi moraju
zadovoljavati su pouzdanost i predvidivost u izvršavanju zadataka.
Zbog navedenog svojstva, sustavi za rad u stvarnom vremenu (u daljnjem tekstu SRSV) neizostavan su dio svih ugradbenih 
računalnih sustava u kojima postoje kritični poslovi čije neizvršavanje ili prekasno izvršavanje izaziva katastrofalne
posljedice ili znatnu degradaciju performansi. Primjeri takvih sustava su automobili i ostala prometna sredstva, vojna industrija, 
multimedijski sustavi, sustavi automatskog upravljanja, roboti itd. 

Ponekad SRSV zbog iznenadne pojave dodatnih zadataka mogu ući u stanje preopterećenja. To je stanje u kojemu procesor ne može 
izvršiti sve instance zadataka na vrijeme i neke od njih mora preskočiti. Međutim, u nekim primjenama preskakanje instanci zadataka ne smije biti nasumično, nego
mora biti kontrolirano i predvidivo. Kod sustava automatskog upravljanja preskakanje kritičnih zadataka može sustav dovesti do nestabilnosti dok u 
multimedijskim sustavima izaziva smanjenje performansi.
  
Za primjer možemo uzeti autonomni mobilni robot koji ima sustav za izbjegavanje prepreka. Za siguran rad ključno je uspješno izvođenje zadataka
vezanih uz lokalizaciju i kontrolu motora. Ako takav robot uđe u stanje preopterećenja, prioritet 
moraju dobiti zadatci koji su vezani uz detekciju prepreke i kontrolu motora kako bi se robot na vrijeme zaustavio ukoliko je to potrebno.
Ostale funkcije koje robot obavlja manje su kritične te se mogu propustiti bez štete za sustav.

U ovom radu razmatraju se rješenja koja bi osigurala determinističko ponašanje SRSV-a u uvjetima preopterećenja. Što je instanca zadatka, tzv. posao, kritičniji,
to treba dobiti veći prioritet pri raspoređivanju. Tako se izbjegava šteta koja bi potencijalno nastala propuštanjem takvih zadataka. 
SRSV-i koji su najčešće zastupljeni u industrijskim primjenama obično ne osiguravaju sigurnost sustava kada je on u stanju preopterećenja. Stoga je ideja ovog rada 
modificirati SRSV otvorenog koda, konkretno FreeRTOS, kako bi se ugradio mehanizam za smanjenje trajnog preopterećenja kontroliranim 
propuštanjem nekritičnih poslova. Točnije,  
modifikacijom jezgre FreeRTOS-a implementiran je strogi sustav za rad u stvarnom vremenu s ublaženim uvjetima (engl. \textit{weakly hard real-time system}). Kod takvog sustava
povremeno se dopušta da se posao ne izvede, ali u određenom broju slijedno aktiviranih poslova postoji striktno definirana donja granica na broj 
poslova koji se moraju pravovremeno izvesti.
Takvim pristupom preskakanja poslova osigurava se da se niti jedan zadatak trajno ne blokira. Nadalje implementirana je strategija 
prekidanja zadataka, prema kojoj se zadatci ne nastavljaju izvršavati nakon što su došli do krajnjeg roka za izvršavanje.

U literaturi se može pronaći mnogo različitih algoritama za raspoređivanje zadataka, s obzirom na optimalnost, složenost i model zadataka na koji se odnose. 
Neki od njih implementirani su u raspoređivač zadataka FreeRTOS-a 
te međusobno uspoređivani nad istim skupovima zadataka. Cilj je pronaći algoritam koji će u uvjetima preopterećenja osigurati 
determinističko ponašanje uz što veću kvalitetu usluge, to jest uz što više pravovremeno izvršenih poslova.

\end{document}