\documentclass[../zavrsni.tex]{subfiles}

\begin{document}

Operacijski sustavi za rad u stvarnom vremenu (eng. Real Time Operating systems) sustavi su u kojima nije bitan samo
rezultat operacije, nego je jednako važna i pravovremenost njezina izvršenja. Najvažnija svojstva koja takvi sustavi moraju
zadovoljavati su pouzdanost i predvidivost u izvršavanju zadataka.
%Za primjer možemo uzeti zbrajanje dva broja.. ma ipak ne
Zbog navedenog svojstva sustavi za rad u stvarnom vremenu (u daljnjem tekstu SRSV) neizostavni su dio svih ugradbenih 
računalnih sustava u kojima postoje kritični poslovi čije neizvršavanje ili prekasno izvršavanje izaziva katastrofalne
posljedice. Primjeri takvih sustava su automobili i ostala prometna sretstva,vojna industrija, multimedijski sustavi, roboti itd. 
Ponekad SRSV zbog iznenadne pojave dodatnih zadataka mogu ući u stanje preopterećenja. To je stanje u kojemu procesor ne može 
izršiti sve poslove na vrijeme i neke od njih mora preskočiti. Međutim, preskakanje zadataka ne smije biti nasumično, nego
mora biti kontrolirano i predvidivo. 

Za primjer možemo uzeti robota koji ima sustav za izbjegavanje prepreka. (OVO DODAJ !!!!)

Potrebno je razmotriti rješenja koja će osigurati da najkritičniji poslovi dobiju veći prioritet za izvršavanje. 
Na taj način se izbjegava šteta koja bi potencijalno nastala propuštanjem takvih zadataka. 

Komercijalno dostupni SRSV-ovi ne osiguravaju sigurnost sustava kada on upadne u stanje preopterećenja. Stoga je ideja ovog rada 
modificirati SRSV otvorenog koda, za što je odabran FreeRTOS, kako bi se poslovi u stanju trajnog preopterećenja mogli kontrolirano propuštati. 
Modifikacijom jezgre FreeRTOS-a implementiran je strogi sustav za rad u stvarnom vremenu sa ublaženim uvjetima (eng. weakly hard). Kod takvog sustava
povremeno dopuštamo da se posao ne izvede, ali na način da se unutar određenog broja slijednih poslova samo jedan smije propustiti.
Takvim pristupom se osigurava da nikad ne dođe do blokade pojedinog zadatka te da se njegovo izvršavanje uvijek preskače.

U literaturi se može pronaći mnogo algoritama za raspoređivanje zadataka. Neki od njih bit će implementirani i kasnije uspoređivani na istim
setovima zadataka. Cilj je pronaći algoritam koji će optimalno rasporediti zadatke u uvjetima preopterećenja, to jest kod kojeg će biti 
zadovoljeni svi uvjeti strogog SRVS-a s ublaženim uvjetima uz što manje propuštenih poslova. 


\end{document}