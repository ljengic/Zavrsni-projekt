\documentclass[../zavrsni.tex]{subfiles}

\begin{document}

Operacijski sustavi za rad u stvarnom vremenu (eng. Real Time Operating systems) sustavi su u kojima nije bitan samo
rezultat operacije, nego je jednako važna i pravovremenost njezina izvršenja. Najvažnija svojstva koja takvi sustavi moraju
zadovoljavati su pouzdanost i predvidivost u izvršavanju zadataka.
Zbog navedenog svojstva sustavi za rad u stvarnom vremenu (u daljnjem tekstu SRSV) neizostavni su dio svih ugradbenih 
računalnih sustava u kojima postoje kritični poslovi čije neizvršavanje ili prekasno izvršavanje izaziva katastrofalne
posljedice. Primjeri takvih sustava su automobili i ostala prometna sredstva,vojna industrija, multimedijski sustavi, roboti itd. 

Ponekad SRSV zbog iznenadne pojave dodatnih zadataka mogu ući u stanje preopterećenja. To je stanje u kojemu procesor ne može 
izvršiti sve poslove na vrijeme i neke od njih mora preskočiti. Međutim, preskakanje zadataka ne smije biti nasumično, nego
mora biti kontrolirano i predvidivo. 
Za primjer možemo uzeti robota koji ima sustav za izbjegavanje prepreka. Ako takav robot uđe u stanje preopterećenja, prioritet 
moraju dobiti zadatci koji su vezani uz detekciju prepreke i kontrolu motora kako bi se robot na vrijeme zaustavio ukoliko je to potrebno.
Ostale funkcije koje robot obavlja manje su kritične te se mogu propustiti bez štete za sustav.

Potrebno je razmotriti rješenja koja bi osigurala determinističko ponašanje SRSV-a u uvjetima preopterećenja.Što je posao kritičniji,
to treba dobiti veći prioritet pri raspoređivanju. Tako se izbjegava šteta koja bi potencijalno nastala propuštanjem takvih zadataka. 
Komercijalno dostupni SRSV-ovi ne osiguravaju sigurnost sustava kada je on u stanju preopterećenja. Stoga je ideja ovog rada 
modificirati SRSV otvorenog koda, za što je odabran FreeRTOS, kako bi se poslovi u stanju trajnog preopterećenja mogli kontrolirano propuštati. 
Modifikacijom jezgre FreeRTOS-a implementiran je strogi sustav za rad u stvarnom vremenu s ublaženim uvjetima (eng. weakly hard). Kod takvog sustava
povremeno dopuštamo da se posao ne izvede, ali na način da se unutar određenog broja slijednih poslova samo jedan smije propustiti.
Takvim pristupom preskakanja zadataka osigurava se da se niti jedan zadatak trajno ne blokira. Nadalje implementirati će se strategija 
prekidanja zadataka, prema kojoj se zadatci ne nastavljaju izvršavati nakon što su došli do krajnjeg roka za izvršavanje.

U literaturi se može pronaći mnogo algoritama za raspoređivanje zadataka. Neki od njih bit će implementirani u raspoređivač zadataka FreeRTOS-a 
te kasnije uspoređivani nad istim setovima zadataka. Cilj je pronaći algoritam koji će u uvjetima preopterećenja osigurati 
determinističko ponašanje uz što veću kvalitetu usluge, to jest uz što više izvršenih poslova do njihovog krajnjeg roka završetka


\end{document}