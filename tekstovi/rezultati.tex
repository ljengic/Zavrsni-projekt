\documentclass[../zavrsni.tex]{subfiles}

\begin{document}

Postupkom opisanim u prošlom poglavlju generirani su izvještaji za ranije opisane algoritme koji su implementirani unutar FreeRTOS-a.
OPIŠI KAKO SDE DOBIJU GRAFOVI , PYTHON ITD. (ili to ipak u prethodno poglavlje hmm) 

Za početak će biti prikazani rezultati simulacija algoritma EDF. Graf koji prikazuje kvalitetu usluge u odnosu na faktor opterećenja 
prikazan je na slici 5.1. 
\begin{figure}[!htb]
    \center{\includegraphics[width=\textwidth]
    {images/Figure_3.png}}
    \caption{\label{fig:my-label} Usporedba algoritama BWP i RTO}
\end{figure}
Algoritam EDF je izdvojen od ostala dva jer nije primjenjiv za ublaženo-stroge
uvjete u sustavima za rad u stvarnom vremenu. Stoga je uz kvalitetu usluge, na posebnom grafu na slici N.M prikazan odnos broja kršenja 
postavljenih uvjeta u skip over modelu i ukupnog broja zadataka. Kršenje ublaženo-strogih uvjeta prikazuje se relativno u odnosu na broj 
poslova jer svaka simulacija ima različite vrijednosti perioda, a time i različit broj poslova.
Vidljivo je kako opterećenje sustava raste, sve više puta se krše uvjeti postavljeni nad 
skupom zadataka.

OVDJE IDE JOS JEDAN GRAF, podatci su generirani samo ga treba plotati.

U kontekstu skip-over modela ublaženo-strogih uvjeta potrebno je usporediti algoritme RTO i BWP. Graf dobiven iz podataka skupljenih simulacijama
prikazan je na slici 5.3. Na grafu su prikazane kvalitete usluga u ovisnosti o faktoru opterećenja. Vidljivo je da BWP algoritam za mala 
opterećenja ima puno veću kvalitetu usluge, koja opada kako opterećenje sustava raste. Kod algoritma RTO kvaliteta usluge je gotovo nepromjenjiva
u odnosu na faktor opterećenja i znatno niža od one kod algoritma BWP. 

\begin{figure}[!htb]
    \center{\includegraphics[width=\textwidth]
    {images/Figure_2.png}}
    \caption{\label{fig:my-label} Usporedba algoritama BWP i RTO}
\end{figure}

\end{document}

jos pokoja pametna recenica...