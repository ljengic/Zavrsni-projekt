\documentclass[../zavrsni.tex]{subfiles}

\begin{document}

\section{Generiranje setova testnih zadataka}

%\subsection{UUniFast algoritam}

%\subsection{Generiranje vrijednosti perioda}

%\subsection{Implementacija generatora testnih zadataka}

Generiranje seta zadataka implementirano je u taskSetGenerator.c datoteci. U strukturi periodic\_task sadržane su sve 
informacije potrebne za opis pojedinog zadatka, kontrolu njegovog izvođenja te prikupljanje podataka o poštivanju rokova 
izvršavanja tijekom simulacije.
\begin{lstlisting}[style=CStyle,caption={Struktura periodic\_task},captionpos=b]
struct periodic_task{
    TaskHandle_t handler;
    char * name;
    double u;
    TickType_t period;
    TickType_t duration;
    int weakly_hard_constraint;
    int numOfPeriods;
    bool report[MAX_PERIOD_CNT];
    int missed_deadlines;
    int times_killed;
} Task_Set[MAX_TASK_CNT];
\end{lstlisting}
Za svaki zadatak definirana je jedna inačica ove strukture. Infomacije o periodima i weakly hard ograničenjima zadataka 
zadane su u tekstuanim datotekama i ranije generirane(OPISI TO KAD NAPRAVIS).
Ti podatci se procitaju iz datoteka i upišu u odgovarajuće varijable strukture pojedinog zadatka.
Za generiranje faktora opterećenja za set testnih zadataka korišten je UUniFast algoritam. Algoritam prima ukupan broj zadataka 
i sumu faktora opterećenja svih zadataka te uniformno raspoređuje faktore opterećenja. Vremenska složenost algoritma je O(n).
Vrijeme izvršavanja pojedinog zadatka dobiveno je kao umnožak perioda i faktora opterećenja.
Imena zadataka generiraju se u obliku Task\_xx, gdje xx predstavlja id pojedinog zadatka, počevši od 1 do broja zadataka.

Prije početka svake simulacije poziva se funkcija startTaskSetGenerator koja poziva potrebne funkcije kako bi se izvršili 
svi ranije navedeni koraci. Navedena funkcija kao argumente prima ukupan broj zadataka koje je potrebno generirati, zadani 
faktor opterećenja i putanju do datoteke u koju će se pohraniti rezultati simulacije.
\begin{lstlisting}[style=CStyle,caption={Funckija startTaskSetGenerator},captionpos=b]
void startTaskSetGenerator(double utilization,int n, char * report_file){
    TASK_CNT = n; 
    total_utilization = utilization; 
    file_path = report_file;
    calculateUtilization(utilization);
    readTaskPeriods();
    readWeaklyHardConstraint();
    calculateTaskDuration();
    calculateHiperperiod();
    calculateNumOfPeriods();
    generateTaskNames();
    resetTimesKilled();
    resetReports();
}
\end{lstlisting}
\section{Tijek simulacije}

Tijekom izvođenja simulacije za svaki zadatak se pamti je li izvršen do roka završetka u svakom periodu. 
Na taj način se dobije informacija o izvođenju svakog zadatka u svakom vremenskom intervalu perioda. To je implementirano 
pomoću bool niza čije su vrijednosti na početku simulacije postavljene na 0 i nakon što se zadatak uspješno izvede 
prije roka završetka ta vrijednost postavi se u 1. Simulacija se prekida nakon hiperperioda (najmanjeg zajedničkog višekratnika perioda 
svih zadataka).Nakon završetka simulacije, na temelju ranije navedenih podataka, za svaki zadatak se zbraja broj propuštenih rokova završetaka.
te se u csv (eng. comma-separated values) datoteku izvještaja dodaje novi redak koji opisuje trenutno provedenu simulaciju. 
Podatci koje je bitno upisati u izvještaj su ukupni broj zadataka koji se nisu izvršili do roka završetka, kao i broj propuštenih rokova 
završetka svakog pojedinog zadatka. Također je bitno znati jesu li bili zadovoljeni svi weakly hard uvjeti 
postavljeni nad simulacijom te kolika je stopa uspješnosti izvršenja zadataka (eng. Quality of Service). 

\section{Pokretanje simulacije}

Radi usporedbe različitih algoritama nad generiranim setom zadataka potrebno je pokrenuti program uz različite faktore opterećenja. 
Za to je implementirana bash skripa kojom se najprije stvara csv datoteka u kojoj će biti pohranjen izvještaj sa podatcima o 
svim simulacijama pokrenutim za određeni set testnih zadataka. Za stvaranje datoteke izvještaja napisan je poseban c program koji osim što
stvori datoteku zapiše u nju prvi red sa nazivima pojedinih stupaca.
Skripta nadalje pokreće simulaciju od faktora opterećenja 0.9 do 1.5, uz korake 0.01 te se nakon svakog pokretanja simulatora u 
datoteku izvještaja upisuje jedna linija koja sadrži ranije opisane podatke.
Pri pokretanju simulacije programu se preko argumenata prosljeđuju 3 parametra, broj zadataka, 
faktor opterećenja te datoteka na čiji će se kraj upisivati izvještaj nakon završene simulacije.
Faktor opterećenja se radi jednostavnosti implementacije prosljeđuje pomožen sa 100, jer bash ne podržava rad sa decimalnim brojevima.
U nastavku je priložena slika datoteke izvještaja.

DODAJ SLIKU IZVJEŠTAJA !!!

OPISI KAKO SE IZ IZVJEŠTAJA DOBIJU GRAFOVI!!!

IZNOS TICKA NA KONKRETNOM SIMULATORU

\end{document}