\documentclass[../zavrsni.tex]{subfiles}

\begin{document}

\section{Generiranje skupova testnih zadataka}

%\subsection{UUniFast algoritam}

%\subsection{Generiranje vrijednosti perioda}

%\subsection{Implementacija generatora testnih zadataka}

Generiranje skupova zadataka implementirano je u datoteci \texttt{taskSetGenerator.c}. U strukturi \texttt{periodic\_task} sadržane su sve 
informacije potrebne za opis pojedinog zadatka, kontrolu njegovog izvođenja te prikupljanje podataka o poštivanju rokova 
izvršavanja tijekom simulacije. Za svaki zadatak definirana je jedna inačica ove strukture. 
\begin{lstlisting}[style=CStyle,caption={Struktura periodic\_task},captionpos=b]
struct periodic_task{
    TaskHandle_t handler;
    char * name;
    double u;
    TickType_t period;
    TickType_t duration;
    int weakly_hard_constraint;
    int numOfPeriods;
    bool report[MAX_PERIOD_CNT];
    int missed_deadlines;
    int times_killed;
} Task_Set[MAX_TASK_CNT];
\end{lstlisting}

Vrijednosti perioda generirane su uniformnom razdiobom. Gornja i donja granica intervala iz kojeg se nasumično biraju vrijednosti zadane su konstantama.
U konkretnom slučaju provedenih simulacija vrijednosti perioda su iz intervala [20,100]. Male vrijednosti izbjegnute su zbog nepreciznosti kod zaokruživanja pri
računanju vremena izvršavanja, što u najgorem slučaju ima za posljedicu znatnu promjenu faktora opterećenja sustava. 
Sljedeća veličina koja je potrebna za 
provedbu simulacije je hiperperiod. To je najmanji zajednički višekratnik vrijednosti perioda svih zadataka. Nakon vremena hiperperioda, raspored poslova 
se ponavlja jer se krajnji rokovi završetka svih zadataka poklope. Zbog toga je simulaciju potrebno provesti od trenutka t=0 do vrijendosti hiperperioda. Vrijednosti hiperperioda mogu biti prevelike te 
bi zbog toga simulacije trajale predugo. Kako bi se spriječio opisani problem, nakon generiranja perioda računa se vrijesnot hiperperioda i ukoliko je veća od zadane konstante
 generiranje perioda se pokreće iznova. Konstanta je izračunata tako da se simulacija ne izvodi više od 10 sekundi.

Za generiranje faktora opterećenja za skup testnih zadataka korišten je algoritam UUniFast. Algoritam prima ukupan broj zadataka 
i sumu faktora opterećenja svih zadataka te uniformno raspoređuje faktore opterećenja. Ovaj algoritam korišten je jer pri generiranju nasumičnih
vrijednosti ne daje prednost niti jednoj vrijednosti. Vremenska složenost algoritma je O(n). Kako bi generirani skup zadataka bio što sličniji stvarnim uvjetima
, nakon generiranja faktora opterećenja ugrađena je provjera da neki zadatak ne zauzima previše procesorskog vremena. Konkretno, ako neki zadatak ima faktor
opterećenja veći od 0.75, sve vrijednosti se odbaciju i algoritam se ponavlja. Ovu provjeru bilo je potrebno napraviti i iz razloga što je moguće da faktor
opterećenja zadatka bude veći od 1, što nema smisla razmatrati.

Vrijeme izvršavanja pojedinog zadatka dobiveno je kao umnožak perioda i faktora opterećenja.
Imena zadataka generiraju se u obliku \texttt{Task\_xx}, gdje xx predstavlja id pojedinog zadatka, počevši od 1 do broja zadataka.

Ublaženo-strogi uvjeti zadataka, kao i periodi, generirani su uniformnom razdiobom. Vrijednosti su iz intervala [0,5]. DODAJ OPIS ZA UVJETE RASPOREDIVOSTI

Prije početka svake simulacije poziva se funkcija \texttt{startTaskSetGenerator()} koja poziva potrebne funkcije kako bi se izvršili 
svi ranije navedeni koraci za generiranje zadataka. Navedena funkcija kao argumente prima ukupan broj zadataka koje je potrebno generirati, zadani 
faktor opterećenja i putanju do datoteke u koju će se pohraniti rezultati simulacije.
\begin{lstlisting}[style=CStyle,caption={Funckija startTaskSetGenerator},captionpos=b]
void startTaskSetGenerator(double utilization,int n, char * report_file){
    TASK_CNT = n; 
    total_utilization = utilization; 
    file_path = report_file;
    calculateUtilization(utilization);
    readTaskPeriods();
    readWeaklyHardConstraint();
    calculateTaskDuration();
    calculateHiperperiod();
    calculateNumOfPeriods();
    generateTaskNames();
    resetTimesKilled();
    resetReports();
}
\end{lstlisting}
\section{Tijek simulacije}

Tijekom izvođenja simulacije za svaki se zadatak pamti je li izvršen do roka završetka u svakom periodu. 
Na taj način se dobije informacija o izvođenju svakog zadatka u svakom pripadajućem periodu. To je implementirano 
pomoću polja varijabli tipa bool čije su vrijednosti na početku simulacije postavljene na 0 i nakon što se posao pravovremeno izvede 
ta vrijednost se postavi u 1. Simulacija se prekida nakon hiperperioda (najmanjeg zajedničkog višekratnika perioda 
svih zadataka).

DODAJ SLIKU JEDINICA I NULA ONIH SILNIH !!!

Iz polja nula i jedinica, prikazanog na slici x.y, koje predstavljaju izvršavanje poslova, potrebno je interpertirati podatke o provedenoj simulaciji.
Za svaki zadatak potrebno je pobrojiti propuštene rokove završetka, kao i broj kršenja ublaženo strogih ujeta postavljenih nad generiranim skupom.
Također je bitno znati jesu li bili zadovoljeni svi ublaženo-strogi uvjeti 
postavljeni nad skupom zadataka te kolika je kvaliteta usluge. Za sve navedeno implementirane su funkcije u datoteci \texttt{taskSetGenerator.c}.

OPISI CSV DATOTEKU
te se u csv (eng. comma-separated values) datoteku izvještaja dodaje novi redak koji opisuje trenutno provedenu simulaciju. 

\section{Pokretanje simulacije}

Radi usporedbe različitih algoritama nad generiranim setom zadataka potrebno je pokrenuti program uz različite faktore opterećenja. 
Za to je implementirana bash skripa kojom se najprije stvara csv datoteka u kojoj će biti pohranjen izvještaj sa podatcima o 
svim simulacijama pokrenutim za određeni set testnih zadataka. Za stvaranje datoteke izvještaja napisan je poseban c program koji osim što
stvori datoteku zapiše u nju prvi red sa nazivima pojedinih stupaca.
Skripta nadalje pokreće simulaciju od faktora opterećenja 0.9 do 1.5, uz korake 0.01 te se nakon svakog pokretanja simulatora u 
datoteku izvještaja upisuje jedna linija koja sadrži ranije opisane podatke.
Pri pokretanju simulacije programu se preko argumenata prosljeđuju 3 parametra, broj zadataka, 
faktor opterećenja te datoteka na čiji će se kraj upisivati izvještaj nakon završene simulacije.
Faktor opterećenja se radi jednostavnosti implementacije prosljeđuje pomožen sa 100, jer bash ne podržava rad sa decimalnim brojevima.
U nastavku je priložena slika datoteke izvještaja.

DODAJ SLIKU IZVJEŠTAJA !!!

OPISI KAKO SE IZ IZVJEŠTAJA DOBIJU GRAFOVI!!!

IZNOS TICKA NA KONKRETNOM SIMULATORU

\end{document}