\documentclass[../zavrsni.tex]{subfiles}

\begin{document}

\section{Sustavi za rad u stvarnom vremenu}

Sustavi za rad u stvarnom vremenu (eng. Real Time System) danas su neizostavan dio mnogih sustava korištenih u svim
granama ljudske djelatnosti. Kod njih nam nije bitan samo rezultat izvođenja operacije, nego je jednako važno 
i vrijeme u kojem se ta operacija izvede. Zbog toga se svakom zadatku pridjeljuje rok do kojeg se mora izvršiti.
Kako bi sustav radio pouzdano mora se osigurati predvidivo raspoređivanje zadataka koje će osigurati da se što više 
zadataka izvrši na vrijeme.
SRSV mora imati implementiranu kontrolu zadataka, kontrolu vremena, raspoređivač zadataka i sustav komunikacije i sinkronizacije
među zadatcima.

S obzirom na posljedice koje izaziva propuštanje roka izvršavanjazadatke djelimo u dvije skupine :
\begin{itemize}
    \item[--] Kritični zadatci (eng. hard real time tasks)
    \item[--] Zadatci koji se mogu preskočiti (eng. soft real time tasks)
\end{itemize}
Kritični zadatci niti jednom ne smiju propustiti krajnji rok izvršavanja jer su to zadatci koji obavljaju važan posao. Njihovo 
propuštanje izaziva katastrofalne posljedice koje mogu biti pogubne za cijeli sustav. Dobar primjer kritičnog zadatka je detekcija 
pritiska papučice kočnice na automobilu. S druge strane, propuštanje nekih zadataka
nam nije kritično i neće nanjeti štetu sustavu, nego će smanjiti performanse ili mogućnosti sustava. Primjer ovakvog zadatka je 
paljenje signalne ledice.
%preopterećenje ovdje objasni
%QoS objasni

Zadatci u operacijskim sustavima za rad u stvarnom vremenu općenito se mogu podjeliti u 3 grupe :
\begin{itemize}
    \item[--] Zadatak koji se trenutno izvodi (eng. running task)
    \item[--] Zadatci koji su spremi za izvođenje (eng. ready tasks)
    \item[--] Zadatci koji su blokirani i čekaju određeni događaj (eng. blocked tasks)
\end{itemize}

Kod sustava sa jednom procesorskom jezgrom samo jedan zadatak se u određenom trenutku može izvoditi. Zadatci koji su spremni za izvođenje
čekaju oslobođenje procesora i jedan od njih se odabire izvršavanje.
Dio jezgre SRSV-a koji je zadužen za izbor zadatka koji je spreman za izvođenje i koji će se idući izvršavati naziva se raspoređivač 
zadataka (eng. Task Scheduler). To je najvažniji dio jezgre SRSV-a jer o raspoređivanju zadataka ovisi kolika će biti kvaliteta usluge i
koliko pouzdan će biti cijeli sustav.
Neki zadatci mogu biti privremeno ili trajno blokirani i dok su u tom stanju raspoređivač zadataka ih ne uzima u obzir.
Treba napomenuti da je ovo generalna podjela koja se razlikuje u implementaciji konkretnih SRSV-ova.


\section{Periodični zadatci u sustavima za rad u stvarnom vremenu}

Periodični zadatci su zadatci koji se iznova ponavljaju u istim vremenskim intervalima Ti. Taj interval naziva se period zadatka.
Svaki zadatak opisuje i vrijeme njegova izvršavanja Ci (eng. Computation time). Period i vrijeme izvršavanja zajedno daju veličinu 
koju nazivamo faktor opterećenja (eng. Utilization) koja opisuje postotak zauzeća procesora pojedinog zadatka u jednom periodu.
\begin{equation*}
    U\textsubscript{i} = \frac{C\textsubscript{i}}{D\textsubscript{i}}
\end{equation*}
Nadalje, bitna veličina koja opisuje svaki zadatak je krajnji rok njegova završetka (eng. deadline). U ovom radu razmatrani su 
isključivo zadatci čiji krajnji rok završetka je jednak periodu. Kao primjer dan je vremenski dijagram s jednim periodičnim zadatkom.
Period zadatka iznosi 4 vremenske jedinice, a vrijeme izvršavanja pojedinog posla 1 vremensku jedinicu. Pomoću ta dva podatka i ranije 
dane formule proizlazi da je faktor opterećenja zadatka 0.25. Drugim rječima, ovaj zadatak zauzima 25 \% ukupnog procesorskog vremena.
\begin{figure}[h]
    \centering

    \begin{RTGrid}[width=13cm]{1}{20}

      \TaskArrDead{1}{0}{4}     
      \TaskArrDead{1}{4}{4}
      \TaskArrDead{1}{8}{4}
      \TaskArrDead{1}{12}{4}
      \TaskArrDead{1}{16}{4}
  
      \TaskExecution{1}{0}{1}
      \TaskExecution{1}{4}{5}
      \TaskExecution{1}{8}{9}
      \TaskExecution{1}{12}{13}
      \TaskExecution{1}{16}{17}

    \end{RTGrid}

    \caption{Primjer periodičnog zadatka}
    \label{fig:ex1}
  \end{figure}

Ukupno opterećenje sustava dobiva se kao zbroj faktora opterećenja svih zadataka. Ukoliko je sustav preopterećen, tj. ukupno opterećenje mu 
je veće od 1 neki od zadataka se neće izvršiti do svog roka za izvršavanje. U tom slučaju koriste se algoritmi koji će osigurati 
da se zadatci ne preskaču nasumično, već na kontroliran način kao bi se sustav zaštitio od potencijalnih oštećenja nastalih preskakanjem 
kritičnih zadataka. Postoje dvije vrste preopterećenja sustava.
\begin{itemize}
    \item[--] Trajno preopterećenje (eng. Permanent Overload)
    \item[--] Prolazno preopterećenje (eng. Transient Overload)
\end{itemize}
Kod prolaznog opterećenja sustava ima ukupno opterećenje manje ili jednako 1, no u nekom trenutku može doći do aktivacije aperiodičnog zadatka 
koji onda sustav gura u stanje preopterećenja. Nakon što prođe određeno vrijeme i aperiodični zadatak se izvede sustav se vraća u prijašnje
stanje i svi zadatci se izvršavaju do svog krajnjeg roka. Drugi slučaj je kada je sustav u stanju trajnog preopterećenja, kada je ukupno 
opterećenje sustava trajno veće od 1. Tada se svi poslovi neće moći izvršiti i kontinuirano će se pojedini poslovi morati propuštati. 

%treba li dodati vrste weakly hard uvjeta

U ovom radu ispitivat će se ponašanje sustava sa ublaženo-kritičnim uvjetima (eng. weakly hard constraints). Za razliku od 
kritičnih zadataka, kod ublaženo-kritičnih zadataka povremeno dopuštamo da se zadatak ne izvede na vrijeme, ali na predvidiv i 
kontroliran način. Na temelju važnosti pojedinog zadatka za rad cjelokupnog sustava određuje se u koliko slijednih perioda 
se zadatak može jednom propustiti. Ovaj pristup osigurava znatno bolje ponašanje u uvjetima preopterećenja jer na siguran 
način preskaćemo izvođenje pojedinih zadataka.

\section{Algoritmi za raspoređivanje zadataka}

U ovom radu ispitivati će se raličiti algoritmi za raspoređivanje zadataka u sustavu koji se nalazi u stanju trajnog preopterećenja.
To je slučaj kada je suma faktora opterećenja svih zadataka veća od 1, i kada se svi zadatci ne mogu izvršiti do svog krajnjeg roka 
izvršenja. U navedenom slučaju važno je osigurati preskakanje zadataka na predvidiv i za sustav siguran način.

Mjera kojom će se uspoređivati učinkovitost pojedinih algoritama naziva se kvaliteta usluge (eng. Quality of Service).
Računa se kao omjer broj zadataka koji su se izvršili do krajnjeg roka završetka i ukupnopg broja svih zadataka.

% popravi da š bude u formuli
\begin{equation*}
    QoS = \frac{broj\ izvršenih\ zadataka\ do\ krajnjeg\ roka}{ukupan\ broj\ zadataka}
\end{equation*}

\subsection{EDF algoritam}

EDF algoritam (eng. Earliest Deadline First) je algoritam koji pri raspoređivanju zadataka prioritet daje onim zadatcima 
koji imaju raniji rok za završetak. Ukoliko sustav nije preopterećen (ukoliko je ukupni faktor opterećenja manji od 1) 
ovim algoritmom optimalno će se rasporediti zadatci i svi će se izvršiti.

\subsection{RTO algoritam}

RTO algoritam (eng. Red Tasks Only) je algoritam prema kojemu se samo izvršavaju kritični zadatci, 
dok se oni koji se mogu preskočiti uvijek preskaču. Prema ovom algoritmu osigurano je poštivanje zadanih uvjeta, no pri manjim 
faktorima opterećenja ovaj algoritam 
nije optimalan. Razlog tomu je što postoji slobodno procesorsko vrijeme u kojem bi se mogli izvršiti zadatci koje nije nužno izvršiti, no 
oni se automatski izbacuju iz rasporeda. Kritični zadatci raspoređuju se prema ranije opisanom EDF algoritmu. 

\subsection{BWP algoritam}

BWP algoritam (eng. Blue When Possible) je poboljšanje ranije opisanog RTO algoritma. Kod BWP algoritma prioritet imaju kritični zadatci, 
no i zadatci koji se ne moraju nužno izvršiti su u listi čekanja. Na taj način, ukoliko se svi kritični zadatci izvrše na red će doći
i opcionalni zadatci. Ovom modifikacijom znatno se poboljša stopa uspješnosti izvršenja zadataka (eng. Quality of Service), pogotovo pri
 manjim faktorima opterećenja.

\end{document}