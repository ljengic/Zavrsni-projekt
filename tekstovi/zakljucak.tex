\documentclass[../zavrsni.tex]{subfiles}

\begin{document}

\sloppy

\justifying

U radu je opisana modifikacija operacijskog sustava za rad u stvarnom vremenu FreeRTOS koja omogućuje podršku za model strogih zadataka s ublaženim uvjetima (engl. \textit{weakly hard real-time system}). U uvodnom dijelu rada opisani su osnovni pojmovi vezani uz sustave za 
rad u stvarnom vremenu te problemi koji nastaju kada sustav uđe u stanje trajnog preopterećenja. Nadalje, opisane su osnove teorije raspoređivanja 
zadataka s opisom nekoliko jednostavnih algoritama. 
U jezgri operacijskog sustava FreeRTOS implementirana je podrška za periodične zadatke i model SRSV-a s ublaženo-strogim uvjetima.
Također, implementirana je strategija prekidanja poslova koja je dodatno poboljšala rasporedivost poslova.
Za potrebe rada razvijen je i generator testnih zadataka te su različiti algoritmi ispitivani i uspoređivani na velikom broju skupova zadataka.
Podatci prikupljeni simulacijama obrađeni su kako bi se mogli usporediti ispitivani algoritmi.

Rezultati dobiveni provedenim simulacijama očekivani su i u skladu s iznesenim teorijskim razmatranjima~\cite{knjiga_buttazzo}. U uvjetima trajnog preopterećenja, uz korištenje \textit{skip-over} modela,  
algoritam BWP daje najveću kvalitetu usluge uz zadovoljene sve postavljene uvjete. Algoritam osigurava kontrolirano propuštanje poslova te je upotrebljiv 
u kontrolnim aplikacijama u kojima postoje kritični poslovi čije propuštanje mora biti determinističko. Modifikacije jezgre FreeRTOS-a izrađene u ovom radu mogu se koristiti neovisno o sklopovskoj platformi.

\end{document}