\documentclass[../zavrsni.tex]{subfiles}

\begin{document}

\sloppy

\justifying

U ovom radu opisana je modifikacija operacijskog sustava za rad u stvarnom vremenu FreeRTOS. Za početak, predočene su osnove sustava za 
rad u stvarnom vremenu i problem koji nastaje kada sustav uđe u stanje trajnog preopterećenja. Nadalje su opisane osnove teorije raspoređivanja 
zadataka s opisom nekoliko jednostavnih algoritama. 
U operacijskom sustavu FreeRTOS je implementirana podrška za periodične zadatke i model SRSV-a s ublaženo-strogim uvjetima.
Također je implementirana strategija prekidanja poslova koja je dodatno poboljšala rasporedivost poslova.
Za potrebe rada razvijen je generator testnih zadataka te su različiti algoritmi ispitivani i uspoređivani na velikom broju skupova zadataka.
Nadalje, podatci prikupljeni simulacijama su obrađeni kako bi se mogli usporediti isptivani algoritmi.

Rezultati dobiveni provedenim simulacijama očekivani su i u skladu s iznesenom teorijom\cite{knjiga_buttazzo}. U uvjetima trajnog preopterećenja, uz korištenje skip-over modela,  
algoritam BWP daje najveću kvalitetu usluge uz zadovoljene sve postavljene uvjete. Algoritam osigurava kontrolirano propuštanje poslova te je upotrebljiv 
u kontrolnim aplikacijama u kojima postoje kritični poslovi čije propuštanje mora biti determinističko.

\end{document}